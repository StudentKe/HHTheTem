% !Mode:: "TeX:UTF-8"

%%% 此部分需要自行填写: (1) 中文摘要及关键词 (2) 英文摘要及关键词
%%%%%%%%%%%%%%%%%%%%%%%%%%%%%

%%======中文摘要===========================%
\begin{cnabstract}
\begin{center}
	\vspace*{-2cm}
	{\heiti \zihao{-2} \textbf{毕业设计(论文)中文摘要} }
	\vspace*{1cm}
\end{center}
\begin{tabular}{|p{15cm}|}% 通过添加 | 来表示是否需要绘制竖线
\hline  % 在表格最上方绘制横线



%%======在下方趋于书写中文摘要=========%%
\vspace*{1cm}
\begin{center}

%%======题目=========%%
	\heiti \zihao{4} {淮海工学院本科毕业论文(设计)~\LaTeX~模板 }
\end{center}
\vspace*{1cm}



\setlength{\parindent}{2em} \textbf{\heiti 摘 \ \ 要:}
%%======下方区域书写摘要正文=========%%
LaTeX(/ˈlɑːtɛx/,常被读作/ˈlɑːtɛk/或/ˈleɪtɛk/),排版时通常使用LATEX,是一种基于TeX的排版系统,由美国计算机科学家莱斯利·兰伯特在20世纪80年代初期开发,利用这种格式系统的处理,即使用户没有排版和程序设计的知识也可以充分发挥由TeX所提供的强大功能,不必一一亲自去设计或校对,能在几天,甚至几小时内生成很多具有书籍质量的印刷品。对于生成复杂表格和数学公式,这一点表现得尤为突出。因此它非常适用于生成高印刷质量的科技和数学、物理文档。这个系统同样适用于生成从简单的信件到完整书籍的所有其他种类的文档。
%%======上方区域书写摘要正文=========%%
\\ 




\vspace*{1cm}
%%%%--  关键词 -----------------------------------------%%%%%%%%
%%%%-- 注意: 每个关键词之间用“;”分开,最后一个关键词不打标点符号
\setlength{\parindent}{2em} \cnkeywords{毕业论文; \LaTeX{}; 模板  }




%%======在上方趋于书写中文摘要=========%%



%%======!! 表格排版实在是迷 请注意调整表格宽度 !!=========%%
\\[10.5cm]
\hline % 在表格最下方绘制横线
\end{tabular}
\end{cnabstract}
\par
\vspace*{2em}





%%====英文摘要==========================%

\begin{cnabstract}
	\begin{center}
		\vspace*{-2cm}
		{\heiti \zihao{-2} \textbf{毕业设计(论文)英文摘要} }
		\vspace*{1cm}
	\end{center}
	\begin{tabular}{|p{15cm}|}% 通过添加 | 来表示是否需要绘制竖线
	\hline  % 在表格最上方绘制横线
	
	
	
	%%======在下方趋于书写英文摘要=========%%
	\vspace*{1cm}
	\begin{center}
	
	%%======题目=========%%
		\heiti \zihao{4} \textbf { Huaihai Institute of Technology Undergraduate thesis (design)~\LaTeX~ template }
	\end{center}
	\vspace*{1cm}
	
	
	
	\setlength{\parindent}{2em} \textbf{\heiti Abstract:}
	%%======下方区域书写摘要正文=========%%
	LaTeX(/ˈlɑːtɛx/,常被读作/ˈlɑːtɛk/或/ˈleɪtɛk/),排版时通常使用LATEX,是一种基于TeX的排版系统,由美国计算机科学家莱斯利·兰伯特在20世纪80年代初期开发,利用这种格式系统的处理,即使用户没有排版和程序设计的知识也可以充分发挥由TeX所提供的强大功能,不必一一亲自去设计或校对,能在几天,甚至几小时内生成很多具有书籍质量的印刷品。对于生成复杂表格和数学公式,这一点表现得尤为突出。因此它非常适用于生成高印刷质量的科技和数学、物理文档。这个系统同样适用于生成从简单的信件到完整书籍的所有其他种类的文档。
	%%======上方区域书写摘要正文=========%%
	\\ 
	
	
	
	
	\vspace*{1cm}
	%%%%--  关键词 -----------------------------------------%%%%%%%%
	%%%%-- 注意: 每个关键词之间用“;”分开,最后一个关键词不打标点符号
	\setlength{\parindent}{2em} \enkeywords{Graduation thesis ; \LaTeX{}; template  }
	
	
	
	
	%%======在上方趋于书写英文摘要=========%%
	
	
	
	%%======!! 表格排版实在是迷 请注意调整表格宽度 !!=========%%
	\\[10cm]
	\hline % 在表格最下方绘制横线
	\end{tabular}
	\end{cnabstract}
	\par
	\vspace*{2em}
	
	
	
	